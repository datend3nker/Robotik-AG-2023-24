\documentclass{article}
\usepackage{listings}
\usepackage{pgfplots}
\usepackage{graphicx}
\usepackage{amsmath}
\usepackage[most]{tcolorbox}
\usetikzlibrary{positioning, arrows.meta}

\pgfplotsset{compat=1.17}

\begin{document}

\title{Python Temperature Conversion Functions}
\author{Robotik AG}
\date{\today}
\maketitle

\section*{Funktionen}

\subsection*{Erklärung}

Die Struktur einer Python-Funktion mit Docstring und Doctest sieht folgendermaßen aus:

\begin{lstlisting}[language=Python]
def function_name(parameter: parameter_type) -> return_type:
    """
    Description of the function.

    Args:
        parameter (parameter_type): Description of the parameter.

    Returns:
        return_type: Description of the return value.

    Raises:
        ExceptionType: Description of the exception (if any).

    Examples:
        >>> function_name(example_value)
        expected_result
    """
    # Implementation of the function
    # ...
    pass
\end{lstlisting}

Hier sind die Erklärungen zu den verschiedenen Teilen:

\begin{itemize}
    \item \textbf{function\_name:} Der Name der Funktion.
    \item \textbf{parameter:} Der Eingabeparameter der Funktion.
    \item \textbf{parameter\_type:} Der Datentyp des Eingabeparameters.
    \item \textbf{return\_type:} Der Datentyp des Rückgabewerts.
    \item \textbf{Description of the function:} Eine kurze Beschreibung der Funktion und ihrer Aufgabe.
    \item \textbf{Description of the parameter:} Eine kurze Beschreibung des Eingabeparameters.
    \item \textbf{Description of the return value:} Eine kurze Beschreibung des Rückgabewerts.
    \item \textbf{Raises:} Eine optionale Sektion, die Ausnahmen dokumentiert, die die Funktion auslösen kann.
    \item \textbf{ExceptionType:} Der Typ der Ausnahme, wenn eine auftritt.
    \item \textbf{Examples:} Beispiele, wie die Funktion verwendet werden kann, zusammen mit den erwarteten Ergebnissen.
\end{itemize}

\subsection*{Doctests: Verwendung und Bedeutung}

\begin{itemize}
    \item \textbf{Doctests} sind spezielle Kommentare in Python-Docstrings, die dazu dienen, Beispiele für die Verwendung der Funktion zu dokumentieren.
    \item \textbf{Vorteile:} Doctests ermöglichen es, die Funktionen automatisch zu testen und sicherzustellen, dass sie wie erwartet funktionieren.
    \item \textbf{Beispiel:} In den \texttt{Examples}-Sektionen sind Beispiele gegeben, wie die Funktion aufgerufen wird und welches Ergebnis erwartet wird.
    \item \textbf{Ausführen:} Doctests können durch Ausführen des Python-Moduls mit dem Parameter \texttt{-m doctest} gestartet werden.
\end{itemize}

\subsection*{Umrechnungsformeln}
\begin{tcolorbox}[colback=red!5!white,colframe=red!75!black]
    Dies kann dir sicher helfen.

    Celsius = 5/9 * (Fahrenheit - 32).

    Celsius = Kelvin - 273.15.

    Die tiefste mögliche Temperatur ist den absoluten Nullpunkt 0K.

  \end{tcolorbox}

\subsection*{Aufgabe 1}
Bitte vervollständige die Funktionen \texttt{celsius\_to\_fahrenheit}, \texttt{fahrenheit\_to\_celsius}, \texttt{celsius\_to\_kelvin}, \texttt{kelvin\_to\_celsius}, \texttt{fahrenheit\_to\_kelvin} und \texttt{kelvin\_to\_fahrenheit}.
Du kannst sie in der Python-Datei \texttt{Temperaturen.py} speichern.
\subsection*{Aufgabe 2}
Schreibe ein Funktion \texttt{main}, die den Benutzer nach eingabe Temperatur fragt und sie dann in die gewünschte Temperatur umrechnet.
 \section{Schleifen}
\subsection*{Aufgabe 1}
Vervollständige die Funktion \texttt{weihnachtsbaum} in \texttt{Schleifen.py} so, dass sie einen Weihnachtsbaum mit der gewünschten Anzahl von Ebenen und Sternen zeichnet.\\
\textbf{Beispiel:}
\begin{lstlisting}[language=Python]
>>weihnachtsbaum(3):
  *
 ***
*****
\end{lstlisting}

\subsection*{Aufgabe 2}
Schreibe jeweils eine Funktion, die folgende Aufgaben erfüllt:
\subsubsection*{a)}
Berechne die Gesamtarbeitsstunden für die Woche.
\subsection*{b)}
Überprüfe, an welchem Tag die meisten Stunden gearbeitet wurden.
\subsection*{c)}
Überprüfe, ob der Schüler an jedem Tag mindestens 6 Stunden gearbeitet hat. Wenn ja, gib eine Nachricht aus, dass die Woche erfolgreich war, sonst eine Nachricht des Bedauerns.
\subsection*{d)}
Erstelle ein neues Dictionary, das die Arbeitsstunden pro Tag in Minuten enthält, wobei angenommen wird, dass 1 Stunde gleich 60 Minuten ist.
\subsection*{e)}
Füge einen neuen Arbeitstag hinzu und weise ihm 8 Stunden zu.
\end{document}

